\documentclass[a4paper, 11pt]{article}
\usepackage[polish]{babel}
\usepackage[MeX]{polski}
\usepackage[utf8]{inputenc}
\usepackage[T1]{fontenc}
%\usepackage{times}
\usepackage{graphicx,wrapfig}
%\usepackage{anysize}
%\usepackage{tikz}
%\usetikzlibrary{calc,through,backgrounds,positioning}
\usepackage{anysize}
\usepackage{float}
%\usepackage{stmaryrd}
%\usepackage{amssymb}
%\usepackage{amsthm}
%\marginsize{3cm}{3cm}{3cm}{3cm}
%\usepackage{amsmath}
%\usepackage{color}
%\usepackage{listings}
%\usepackage{enumerate}
%\lstloadlanguages{Ada,C++}


\begin{document}	
	% \noindent -  w tym akapicie nie bedzie wciecia
	% \ indent - to jest aut., ale powoduje ze jest wciecie
	% \begin{flushleft}, flushright, center - wyrownianie akapitu
	% \textbf{pogrubiany tekst}
	% \textit{kursywa} 
	% 					STRONY 
	%  http://www.codecogs.com/latex/eqneditor.php 
	%  http://www.matematyka.pl/latex.htm
	% 
	\begin{titlepage}
	
	
		
		\newcommand{\HRule}{\rule{\linewidth}{0.5mm}} % Defines a new command for the horizontal lines, change thickness here
		
		\center % Center everything on the page
		
		%----------------------------------------------------------------------------------------
		%	HEADING SECTIONS
		%----------------------------------------------------------------------------------------
		
		\textsc{\LARGE Akademia Górniczo-Hutnicza im. Stanisława Staszica}\\[1.5cm] % Name of your university/college
		\textsc{\Large Kraków}\\[0.5cm] % Major heading such as course name
		\textsc{\large }\\[0.5cm] % Minor heading such as course title
		
		%----------------------------------------------------------------------------------------
		%	TITLE SECTION
		%----------------------------------------------------------------------------------------
		
		\HRule \\[0.4cm]
		{\fontsize{38}{50}\selectfont Symulator pożaru lasu}
	%	{ \Huge \bfseries} Symulator pożaru lasu\\[0.3cm] % Title of your document
		\HRule \\[1.5cm]
		
		%----------------------------------------------------------------------------------------
		%	AUTHOR SECTION
		%----------------------------------------------------------------------------------------
		
		% If you don't want a supervisor, uncomment the two lines below and remove the section above
		\Large \emph{Autorzy:}\\
		Marcin \textsc{Jędrzejczyk}\\ % Your name
		Sebastian \textsc{Katszer}\\ % Your name
		Katarzyna \textsc{Kosiak} \\[3cm]\ % Your name

		
		%----------------------------------------------------------------------------------------
		%	DATE SECTION
		%----------------------------------------------------------------------------------------
		
		{\large \today}\\[3cm] % Date, change the \today to a set date if you want to be precise
		
		%----------------------------------------------------------------------------------------
		%	LOGO SECTION
		%----------------------------------------------------------------------------------------
		
		%\includegraphics{Logo}\\[1cm] % Include a department/university logo - this will require the graphicx package
		
		%----------------------------------------------------------------------------------------
		
		\vfill % Fill the rest of the page with whitespace
		
	\end{titlepage}
	
	%SPIS TRESI
	%
	%
	%
	%
	%
	%
	%
	%
	%
	
	\tableofcontents
	\vfill
	\newpage
	%\pagebreak
	
	%SEKCJE
	%opis zagadnienia, temat, problem, dlaczego chcemy to rozwiązacyać metodą ewolucyjnę
	%jakie są metody rozwiącania problemu, przegląd literatury,
	%proponowane rozwiązania (spójność)
	%czym się inspierowałyśmy
	%
	%
	%
	
	%\setlength{\parskip}{1ex plus 0.5ex minus 0.2ex}
	
	\section{Wstęp}
	\indent
	
	Niniejszy dokument stanowi opis zagadnienia symulowania pożarów lasów wraz z prezentacją symulatora rozprzestrzeniania się pożaru lasu opartego na automatach komórkowych.

	\addcontentsline{toc}{section}{Cele modelowania pożaru}
	\section*{Cele modelowania pożaru}
	 \indent
	 % W sumie to chyba treść tej sekcji ma niewiele wspólnego z jej tytułem. ALe na wiki
	 %omawiane yły fuel models w podrozdziale o takim samym tytule więc też tak zrobiłam
Modelowanie pożaru polega na próbie odtworzenia zachowania się ognia i poznaniu jego parametrów w zadanej sytuacji - m.in. szybkości rozprzestrzeniania się, kierunku i ilości wydzielanego ciepła, estymację skutków pożaru. Na parametry te mają oczywiście wpływ ilość, rodzaj i dokładność dostarczonych danych wejściowch, z których najważniejszym jest rodzaj paliwa. \\
 \indent Istniejące modele paliwowe definiują zestawy cech roślinności mających wpływ na ich palność. Najbardziej znane modele pożaru korzystają z głównych systemów klasyfikacji modeli paliwowych takich jak dynamiczne modele Scotta i Burgana czy trzynaście ``oryginalnych'' modeli paliwowych Andersona i Albiniego, które opisują roślinność w czasie pory suchej, kiedy to stopień zagrożenia pożarowego jest najwyższy. Zwiększa to trafność i przydatność symulacji pożarów podczas organizacji akcji pożarowych.
	 
	
	
	
%	As Anderson states “Fuel models are simply tools to help the user realistically estimate fire behavior. The user must maintain a flexible frame of mind and an adaptive method of operating to totally utilize these aids".[2]
	%napisac o tym ze istnieje wiele modeli it te 13 oryginalnych ze w 
	% dry season bo po to sa symulacje, jakis cytat trzasnac.
	
% wiki Fuel_model <- dużo info na temat paliw
%	- fuel model dla naszego empirycznego, only for DRY SEASON!
%	fala huygensa\\
	
	
	
	
	\addcontentsline{toc}{section}{Czynniki środowiskowe}
	\section*{Czynniki środowiskowe}
	\indent 
	
		Na pożar lasu wpływ mają takie czynniki jak pogoda, charakterystyka paliwa i topografia terenu.	Pogoda wpływa na ogień poprzez kierunek i siłę wiatru i wilgotność powietrza.
		\\ \\TODO trochę szerzej o tym
	\addcontentsline{toc}{section}{Podejścia do modelowania pożaru}
	\section*{Podejścia do modelowania pożaru}
	\indent
	
	Od powstania pierwszych modeli pożarów w latach czterdziestych XX wieku minęło wiele czasu, w ciągu którego zaprezentowano kolejne - zróżnicowane pod względem wymaganych danych wejściowych, znaczących czynników i stopnia rozbudowania - modele.\\
	\indent	Problemem związanym z modelowaniem tak skomplikowanego zjawiska jak ogień jest rosnąca wraz z ilością branych pod uwagę czynników liczba koniecznych do wykoniania obliczeń, a co za tym idzie - potrzeba coraz większej mocy obliczeniowej. Właśnie  z powodu względnie długiego czasu symulacji i potrzeby dużej ilości danych wejściowych skomplikowane modele stosuje się częściej w badaniu niż w terenie.  W związku z tym w istniejących modelach zastosowano różne uproszczenia, często poświęcając mniej znaczące czynniki na rzecz przyspieszenia obliczeń.  \\
\indent	Modele pożaru można podzielić na trzy grupy: empiryczne (model kanadyjski i australijski), semi-empiryczne (automaty komórkowe i Rothermel) i oparte na fizyce (modelowanie ognia koron oraz pełne fizyczne i multifazowe podejście).
	
	\iffalse
	% \iftrue for disabling comment
	 \subsubsection*{Modele empiryczne}
	\begin{itemize}
	\item Kanadyjski
	\item Australijski
	\end{itemize}
	\subsubsection*{Modele semi-empiryczne}
	\begin{itemize}
	\item Automaty komórkowe
	\item Rothermel
	\end{itemize}
	\subsubsection*{Modele oparte na fizyce}
	\begin{itemize}
	\item Modelowanie ognia koron 
	\item Pełne fizyczne i multifazowe podejście
	\end{itemize}
	\fi
	
	\addcontentsline{toc}{section}{Sztandarowe modele}
	\section* {Sztandarowe modele}
	\indent
	
	Poniżej znajduje się krótki przegląd kilku wartych uwagi modeli. Wszystkie wymienione miały znaczącą rolę w rozwoju zagadnienia modelowania pożaru lub są uznawane za najdokładniejsze dla zadanego czasu oczekiwania na rozwiązanie i używane są w najpopularniejszych profesjonalnych programach do symulacji pożaru jak na przykład Farasite, Prometheus czy BEHAVE, które dzięki swojej zdolności do oszacowywania zachowań ognia w czasie rzeczywistym demonstrują wielką użyteczność w terenie.
	\subsection{Rothermel - szybkość rozchodzenia się linii pożaru}
	\indent

	Pierwszy matematyczny model dla symulacji pożaru, został opublikowany w 1972 roku przez Richarda Rothermela.\\
	
	Przybliżone równanie na szybkość rozchodzenia się linii pożaru ma formę:
$$
R=\frac{(I_p)_0(1+\phi_W +\phi_S)}{_b\epsilon Q_{ig}}\\
$$
%% to może kolega mi przetłumaczy lepiej niż google translator
gdzie:\\
$R$	- szybkość rozchodzenia się linii pożaru	[m/min]\\
$(I_P)_0$	- 	strumień ciepła dla warunków bezwietrznych	[kJ/m2/min]\\
$\rho_b$	- 	gęstość drewna całkowicie suchego	[kg/m3]\\
$\epsilon$ 	- efektywność ogrzewania	\\
$Q_{ig}$	- 	ciepło przedzapłonowe	[kJ/kg]\\
$\phi_w$	- 	współczynnik wiatru	\\
$\phi_s$	- 	współczynnik nachylenia\\


\subsection{Rothermel - szybkość rozprzestrzeniania się pożaru w koronach}
\indent

Równanie opisujące szybkość rozprzestrzeniania się pożaru w koronach:
$$
R_{active}=3.34(R_10)_{40\%}
$$
gdzie: \\
$R_{active}$ - szybkość rozprzestrzeniania się pożaru w koronach [m/min]\\
$R_10$ - szybkość rozchodzenia się linii pożaru dla 10. modelu paliwowego i prędkość wiatru na wysokości połowy płomieni równa 40\% prędkości wiatru na wysokości 6,1m. [m/min]\\

\subsection{Van Wagner - intensywność linii ognia}
\indent
Van Wagner zaproponował inne podejście do zagadnienia rozprzestrzenania się pożaru. Równanie opisujące intensywność linii ognia wymaganą do dalszego przeniesienia się ognia:
$$
I`_{initiation}=(\frac{CBH(460+25.9FMC)}{100})^(\frac{3}{2})
$$
gdzie:\\
$I_initation$ - intensywność linii ognia wymaganą do dalszego przeniesienia się ognia [J/m] \\
CBH - podstawowa wysokość roślinności [m]\\
FMC- wilgotność roślinności (podłoża i drzew)\\
%Pozwalam sobie wywalić na razie 
\iffalse
\subsection{Cruz(1999)}
$$
g(x)=\beta_0+\beta_{1}U_10+\beta_{2}FSG+ \sum\limits_{a=1}^{k_j -1}\beta_{ju}D_{ju}+\beta_5EFFM
$$
EFFM- estimated fine ful moisture content(\% ovendry mass basis)\\
$U_10  10-m$ open wind speed
$\beta_1, ...,\beta_4$ -regression coefficients
\fi
\subsection{Cruz - szybkość rozprzestrzeniania się pożaru w koronach}
\indent
Zaproponowane w 2002 roku przez Cruza równanie na szybkość rozprzestrzeniania się pożaru w koronach: 
$$
CROS_A=\beta_1U^{\beta_2}_10 \times CBD^{\beta_3} \times e^{-\beta_4EFFM}
$$
gdzie: \\
$EFFM$ - estymowana wilgotność paliwa \\
$CBD$ - gęstość grupy roślinności [1/m3]\\
$U_10  10-m$ - prędkość wiatru ponad najwyższą roślinnością [m/min]
$\beta_1, ...,\beta_4$ -współczynniki regresji\\
\subsection{Automaty komórkowe}	
\indent
sds
\\TODO BARDZO

	wspomniec o tym z czego korzystaja te systemy firecostam i behaveplus
	
	
	
	\section{Poniższego oprócz bibliografii nie potrzebujemy jeszcze na te konsultacje  Zastosowany model}
	\indent
Automaty komórkowe
		
sds
	\addcontentsline{toc}{section}{Cele symulacji}
	\section*{Cele symulacji}
	\indent
dass

	\addcontentsline{toc}{section}{Dane wejściowe}
	\section*{Dane wejściowe}
	\indent
	ddf
	\addcontentsline{toc}{section}{Sąsiedztwo}
	\section*{Sąsiedztwo}
	\indent	
	
	dsd
	Prędkość wiatru, ukształtowanie terenu. (może wgrywać mapę z wysokościami komórek
	i ona mogłąby być rozszerzana odpowiednio do rozmiaru!).
	\addcontentsline{toc}{section}{Modele paliwowe}
	\section*{Modele paliwowe}
	\indent	
	
xcx
	\addcontentsline{toc}{section}{Cele symulacji2}
	\section*{Cele symulacji2}
	\indent
	
	xcx
	%\addcontentsline{toc}{section}{$ f $ minmaxTransposition}
	\section*{$ f $ fancy function podtytuł}
	\indent
	\begin{figure}[H]%[!htb]
		%		\centerline{\includegraphics[scale=0.7]{Mmin}}
		%	\raggedright{	\caption{pseudokod Mutation.minMaxTransposition}}
	\end{figure}
	\addcontentsline{toc}{section}{Podsumowanie}
	\section*{Podsumowanie}
	\indent
	

	\begin{figure}[bt]
		%		\centerline{\includegraphics[scale=0.7]{alggenmenu}}
		%	\raggedright{	\caption{menu}}
	\end{figure}
	\section{Testy}
	\indent
	sd
	\section{Wnioski}
	\indent
	
	sdsd
	\section{Literatura}
	%\indent
	
\textbf{Asensio MI, Ferragut L., Simon J.:} Modelling of convective phenomena in forest fire. Rev Real Academia de Ciencias, 2002, 96:299–313\\
\textbf{Bodrožić Ljiljana, Stipaniev Darko, Šerić Marijo:} Forest fires spread modeling using cellular automata approach, University of Split, 21000 Split, Croatia, 2009 \\
\textbf{Chad Hoffman:} Fire Behavior Predictions Case Study, University of Idaho, 2007\\
\textbf{Czerpak Tomasz, Maciak Tadeusz:} Modelowanie pożaru lasu. Część 1. Metody i algorytmy modelowania pożaru lasu, Wydział Informatyki, Politechnika Białostocka, 2011 \\
\textbf{Kułakowski Krzysztof:} Automaty Komórkowe, OEN AGH (2000) \\
\textbf{Law A.M., Kelton W.D.:} Simulation Modeling and Analysis, Second Edition, McGraw-Hill 2000\\
\textbf{Ottmar Roger D. et al.:} "An Overview of the Fuel Characteristic Classification System - Quantifying, Classifying, and Creating Fuel beds for Resource Planning." Canadian Journal of Forestry Research. 37:2383-2393. 2007\\
\textbf{Rothermel Richard C.:} A Mathematical Model for Predicting Fire Spread in Wildland Fuels. USDA Forest Service. Research Paper INT-115. 1972.\\
\textbf{Sayama Hiroki:} Introduction to the Modeling and Analysis of Complex Systems, Open SUNY Textbooks, State University of New York at Geneseo, 2015\\	
\textbf{Scott Joe H.,Burgan Robert E.:} Standard Fire Behavior Fuel Models, USDA Forest Service Gen. Tech. Rep. RMRS-GTR-153., June 2005\\	
\textbf{Weise David R., Biging Gregory S.:} A Qualitative Comparison of Fire Spread Models Incorporating Wind and Slope Effects, Research Gate, October 2015\\
\textbf{Wells Gail:} The Rothermel Fire-Spread Study: Still Running Like a Champ, Fire Science Direct, Issue 2, March 2008\\




	%to trzeba zrobić na podstawie zadania z latexa (9?) z odno
\end{document}
