\documentclass[a4paper, 11pt]{article}
\usepackage[polish]{babel}
\usepackage[MeX]{polski}
\usepackage[utf8]{inputenc}
\usepackage[T1]{fontenc}
%\usepackage{times}
\usepackage{graphicx,wrapfig}
%\usepackage{anysize}
%\usepackage{tikz}
%\usetikzlibrary{calc,through,backgrounds,positioning}
\usepackage{anysize}
\usepackage{float}
%\usepackage{stmaryrd}
%\usepackage{amssymb}
%\usepackage{amsthm}
%\marginsize{3cm}{3cm}{3cm}{3cm}
%\usepackage{amsmath}
%\usepackage{color}
%\usepackage{listings}
%\usepackage{enumerate}
%\lstloadlanguages{Ada,C++}


\begin{document}	
	% \noindent -  w tym akapicie nie bedzie wciecia
	% \ indent - to jest aut., ale powoduje ze jest wciecie
	% \begin{flushleft}, flushright, center - wyrownianie akapitu
	% \textbf{pogrubiany tekst}
	% \textit{kursywa} 
	% 					STRONY 
	%  http://www.codecogs.com/latex/eqneditor.php 
	%  http://www.matematyka.pl/latex.htm
	% 
	\begin{titlepage}
	
	
		
		\newcommand{\HRule}{\rule{\linewidth}{0.5mm}} % Defines a new command for the horizontal lines, change thickness here
		
		\center % Center everything on the page
		
		%----------------------------------------------------------------------------------------
		%	HEADING SECTIONS
		%----------------------------------------------------------------------------------------
		
		\textsc{\LARGE Akademia Górniczo-Hutnicza im. Stanisława Staszica}\\[1.5cm] % Name of your university/college
		\textsc{\Large Kraków}\\[0.5cm] % Major heading such as course name
		\textsc{\large }\\[0.5cm] % Minor heading such as course title
		
		%----------------------------------------------------------------------------------------
		%	TITLE SECTION
		%----------------------------------------------------------------------------------------
		
		\HRule \\[0.4cm]
		{\fontsize{38}{50}\selectfont Symulator pożaru lasu}
	%	{ \Huge \bfseries} Symulator pożaru lasu\\[0.3cm] % Title of your document
		\HRule \\[1.5cm]
		
		%----------------------------------------------------------------------------------------
		%	AUTHOR SECTION
		%----------------------------------------------------------------------------------------
		
		% If you don't want a supervisor, uncomment the two lines below and remove the section above
		\Large \emph{Autorzy:}\\
		Marcin \textsc{Jędrzejczyk}\\ % Your name
		Sebastian \textsc{Katszer}\\ % Your name
		Katarzyna \textsc{Kosiak} \\[3cm]\ % Your name

		
		%----------------------------------------------------------------------------------------
		%	DATE SECTION
		%----------------------------------------------------------------------------------------
		
		{\large \today}\\[3cm] % Date, change the \today to a set date if you want to be precise
		
		%----------------------------------------------------------------------------------------
		%	LOGO SECTION
		%----------------------------------------------------------------------------------------
		
		%\includegraphics{Logo}\\[1cm] % Include a department/university logo - this will require the graphicx package
		
		%----------------------------------------------------------------------------------------
		
		\vfill % Fill the rest of the page with whitespace
		
	\end{titlepage}
	
	%SPIS TRESI
	%
	%
	%
	%
	%
	%
	%
	%
	%
	
	\tableofcontents
	\vfill
	%\pagebreak
	
	%SEKCJE
	%opis zagadnienia, temat, problem, dlaczego chcemy to rozwiązacyać metodą ewolucyjnę
	%jakie są metody rozwiącania problemu, przegląd literatury,
	%proponowane rozwiązania (spójność)
	%czym się inspierowałyśmy
	%
	%
	%
	
	%\setlength{\parskip}{1ex plus 0.5ex minus 0.2ex}
	
	\section{Wstęp}
	\indent
	
	Niniejszy dokument stanowi opis zagadnienia symulowania pożarów lasów wraz z prezentacją symulatora rozprzestrzeniania się pożaru lasu.

	\addcontentsline{toc}{section}{Cele modelowania pożaru}
	\section*{Cele modelowania pożaru}
	 \indent
	 
Modelowanie pożaru polega na próbie odtworzenia zachowania się ognia i poznaniu jego parametrów w zadanej sytuacji - m.in. jego szybkość rozprzestrzeniania się, kierunek i ilość wydzielanego ciepła, estymację skutków pożaru. Na parametry te ma oczywiście wpływ ilość, rodzaj i dokładność dostarczonych danych wejściowch, z których najważniejszym jest rodzaj paliwa. \\

 Istniejące modele paliwa definiują zestawy cech roślinności mających wpływ na ich palność. Najbardziej znane modele pożaru korzystają z głównych systemów klasyfikacji modeli paliwa takich jak dynamiczne modele Scotta i Burgana czy trzynaście ``oryginalnych'' modeli paliwa Andersona i Albiniego, które opisują roślinność w czasie pory suchej, kiedy to stopień zagrożenia pożarowego jest najwyższy. Zwiększa to trafność i przydatność symulacji pożarów podczas organizacji akcji pożarowych.
	 
	
	
	
%	As Anderson states “Fuel models are simply tools to help the user realistically estimate fire behavior. The user must maintain a flexible frame of mind and an adaptive method of operating to totally utilize these aids".[2]
	%napisac o tym ze istnieje wiele modeli it te 13 oryginalnych ze w 
	% dry season bo po to sa symulacje, jakis cytat trzasnac.
	
% wiki Fuel_model <- dużo info na temat paliw
%	- fuel model dla naszego empirycznego, only for DRY SEASON!
%	fala huygensa\\
%	automaty gdzies!!!!!!\\
	
	
	
	
	\addcontentsline{toc}{section}{Czynniki środowiskowe}
	\section*{Czynniki środowiskowe}
		Na pożar lasu wpływ mają takie czynniki jak pogoda, charakterystyka paliwa i topografia terenu.\\
		Pogoda wpływa na ogień poprzez wiatr i wilgotność.
	\addcontentsline{toc}{section}{Opis zagadnienia}
	\section*{Podejścia do modelowania pożaru}
	\indent
	
	Od powstania pierwszych modeli pożarów w latach czterdziestych XX wieku minęło wiele czasu, w ciągu którego zaprezentowano kolejne - zróżnicowane pod względem wymaganych danych wejściowych, znaczących czynników i stopnia rozbudowania - modele.\\
	\indent	Problemem związanym z modelowaniem tak skomplikowanego zjawiska jak ogień jest rosnąca wraz z ilością branych pod uwagę czynników liczba koniecznych do wykoniania obliczeń, a co za tym idzie - potrzeba coraz większej mocy obliczeniowej. W związku z tym w istniejących modelach zastosowano różne uproszczenia, często poświęcając mniej znaczące czynniki na rzecz przyspieszenia obliczeń.  \\
	\\
	Modele pożaru można podzielić na trzy grupy: empiryczne, semi-empiryczne i oparte na fizyce.
	\subsubsection*{Modele empiryczne}
	\begin{itemize}
	\item Kanadyjski
	\item Australijski
	\end{itemize}
	\subsubsection*{Modele semi-empiryczne}
	\begin{itemize}
	\item Automat komórkowy
	\item Rothermel
		\begin{itemize}
		\item BEHAVE
		\item SPREAD
		\item FARSITE
		\end{itemize}
	\end{itemize}
	\subsubsection*{Modele oparte na fizyce}
	\begin{itemize}
	\item Modelowanie ognia koron 
	\item Pełne fizyczne i multifazowe podejście
	\end{itemize}
	\addcontentsline{toc}{section}{Popularne modele}
	\section* {Popularne/najważniejsze?/przykładowe modele}
	
	%% co o tym myślicie, chyba za bardzo od rzeczywistości nie odbiegłem
	Farasite i Prometheus używają zależności semi-empirycznych, a także ziemia-korony. Wykorzystują ponadto zasadę rozchodzenia się fal Huygensa.	\\
	Farasite i aplikacje BEHAVE deomstrują wielką użyteczność w terenie, dzięki swojej zdolności do oszacowywania zachowań ognia w czasie rzeczywistym.\\
	Modele skomplikowane testują i pokazują jak prawdopodobnie zachowałby się ogień w różnych scenariusza. Jednak z powodu potrzeby dużej ilości danych wejściowych i czasu symulacji stosuje się je częściej w badaniu niż w terenie. \\
	W terenie używa się modeli, które symulują szybciej, by pozostawić okno czasowe na podjęcie decyzji jak się zachować. 
	
	
	\indent
	TODO\\
	\subsection{Rothermel(1972)}
	\indent
	Pierwszy matematyczny model dla symulacji "wildfires" został opublikowany 1972 przez Rothermela (??i jego zespół ???and several other people who were not named Rothermel). Jego zapis znajduje się poniżej.
	For each fuel particle, the spread rate to other particles is given by: 
$$
R=\frac{(I_p)_0(1+\phi_W +\phi_S)}{_b\epsilon Q_{ig}}\\
$$
%% to może kolega mi przetłumaczy lepiej niż google translator
where\\
$R$	prawie stałe tempo rozprzestrzeniania się stanu	m /minute\\
$(I_P)_0$	stosunek propagacji przepływu bez wiatru ??	kJ / m2 / minute\\
$\rho_b$	przesusznoa masa na gęstość ??	kg / m3\\
$\epsilon$ efektywna liczba ogrzewania	\\
$Q_{ig}$	ciepło przedzapłonu??	kJ / kg\\
$\phi_w$	współczynnik wiatru	\\
$\phi_s$	współczynnik nachylenia\\

\subsection{Rothermel(1991)}
\indent
Crown fire rate of spread
$$
R_{active}=3.34(R_10)_{40\%}
$$
$R_{active}$ the crown fire rate of spread\\
$R_10$ surface fire rate of spread using the fuel characteristics for fuel model 10 and a midflame wind speed set at 40\% the 6.1-m windspeed\\
\subsection{Van Wagner (1977)}
\indent
Crown Fire Initiation Model
$$
I`_{initiation}=(\frac{CBH(460+25.9FMC)}{100})^(\frac{3}{2})
$$
$I_initation$=surface fire line intensity requried for fire transition
Wejściowe zmienne:\\
CBH- Canopy base height\\
FMC- Foliar moisture content
\subsection{Cruz(1999)}
$$
g(x)=\beta_0+\beta_{1}U_10+\beta_{2}FSG+ \sum\limits_{a=1}^{k_j -1}\beta_{ju}D_{ju}+\beta_5EFFM
$$
EFFM- estimated fine ful moisture content(\% ovendry mass basis)\\
$U_10  10-m$ open wind speed
$\beta_1, ...,\beta_4$ -regression coefficients
\subsection{Cruz(2002)}
\indent
Crown fire spread model
$$
CROS_A=\beta_1U^{\beta_2}_10 \times CBD^{\beta_3} \times e^{-\beta_4EFFM}
$$
$EFFM$ estimated fine fuel moisture content (\% ovendry mass bassis)\\
$CBD$ canopy bulk density\\
$U_10  10-m$ open wind speed
$\beta_1, ...,\beta_4$ -regression coefficients\\
	

	wspomniec o tym z czego korzystaja te systemy firecostam i behaveplus
	
	
	
	\section{Poniższego oprócz bibliografii nie potrzebujemy jeszcze na te konsultacje  Zastosowany model}
	\indent

		
sds
	\addcontentsline{toc}{section}{Dane wejściowe}
	\section*{Dane wejściowe}
	\indent
	
	dsd
	\addcontentsline{toc}{section}{Cele symulacji}
	\section*{Cele symulacji}
	\indent
	
xcx
	\addcontentsline{toc}{section}{Cele symulacji2}
	\section*{Cele symulacji2}
	\indent
	
	xcx
	%\addcontentsline{toc}{section}{$ f $ minmaxTransposition}
	\section*{$ f $ fancy function podtytuł}
	\indent
	\begin{figure}[H]%[!htb]
		%		\centerline{\includegraphics[scale=0.7]{Mmin}}
		%	\raggedright{	\caption{pseudokod Mutation.minMaxTransposition}}
	\end{figure}
	\addcontentsline{toc}{section}{Podsumowanie}
	\section*{Podsumowanie}
	\indent
	

	\begin{figure}[bt]
		%		\centerline{\includegraphics[scale=0.7]{alggenmenu}}
		%	\raggedright{	\caption{menu}}
	\end{figure}
	\section{Testy}
	\indent
	sd
	\section{Wnioski}
	\indent
	
	sdsd
	\section{Literatura}
	%\indent
	
\textbf{Asensio MI, Ferragut L., Simon J.:} Modelling of convective phenomena in forest fire. Rev Real Academia de Ciencias, 2002, 96:299–313\\
\textbf{Chad Hoffman:} Fire Behavior Predictions Case Study, University of Idaho, 2007\\
\textbf{Kułakowski Krzysztof:} Automaty Komórkowe, OEN AGH (2000) \\
\textbf{Law A.M., Kelton W.D.:} Simulation Modeling and Analysis, Second Edition, McGraw-Hill 2000\\
\textbf{Ottmar Roger D. et al.:} "An Overview of the Fuel Characteristic Classification System - Quantifying, Classifying, and Creating Fuel beds for Resource Planning." Canadian Journal of Forestry Research. 37:2383-2393. 2007\\
\textbf{Rothermel Richard C.:} A Mathematical Model for Predicting Fire Spread in Wildland Fuels. USDA Forest Service. Research Paper INT-115. 1972.\\
\textbf{Sayama Hiroki:} Introduction to the Modeling and Analysis of Complex Systems, Open SUNY Textbooks, State University of New York at Geneseo, 2015\\	
\textbf{Scott Joe H.,Burgan Robert E.:} Standard Fire Behavior Fuel Models, USDA Forest Service Gen. Tech. Rep. RMRS-GTR-153., June 2005\\	
\textbf{Weise David R., Biging Gregory S.:} A Qualitative Comparison of Fire Spread Models Incorporating Wind and Slope Effects, Research Gate, October 2015\\
\textbf{Wells Gail:} The Rothermel Fire-Spread Study: Still Running Like a Champ, Fire Science Direct, Issue 2, March 2008\\




	%to trzeba zrobić na podstawie zadania z latexa (9?) z odno
\end{document}
